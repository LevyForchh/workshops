\documentclass[]{article}

\usepackage{marginnote}
\usepackage{verbatim}
\usepackage[letterpaper,left=3cm,right=4cm]{geometry}

\usepackage{graphicx}
\usepackage{amsmath}

\usepackage{fancyhdr}
\pagestyle{fancy}

\usepackage[colorlinks=true]{hyperref}
\hypersetup{allcolors={blue}}
\hypersetup{pdftitle={Workshop 6 Homework}}
\hypersetup{pdfauthor={Paul Glezen}}
\hypersetup{pdfcreator={Latex}}
\hypersetup{pdfkeywords={ISAB, Workshop, Probability, Distributions}}

\setlength{\parindent}{0em}
\setlength{\parskip}{1em}

\title{Workshop 6 Homework}
\author{Los Angeles County\\ISAB}
\date{October 18, 2017}

\begin{document}
\maketitle
These exercises are intended to refresh your calculus brain
cells for the upcoming statistics topics.

Print this out and take it with you to your meetings.
Whenever the meeting becomes a waste of your time, discreetly take
these pages out and start filling in the gaps of these derivations.
Meeting hand-outs are a great source of scratch paper.
Feign interest by asking for extra copies.

The gamma function and associated gamma distribution
will be popping up here and there.  It will help to become
familiar with them.

\section*{Normal Distribution}


Evaluate

$$
R = \int_{-\infty}^{\infty} e^{-\frac{1}{2} x^2} dx
$$

\textbf{Trick}: Polar Coordinates

Instead of computing $R$ directly, compute $R^2$ as a
double integral and convert to polar coordinates.  The
symmetry about the origin simplifies things.

\begin{eqnarray*}
R^2 &= & R \cdot R \\
  &= & \int_{-\infty}^{\infty} e^{-\frac{1}{2} x^2} dx 
       \int_{-\infty}^{\infty} e^{-\frac{1}{2} y^2} dy \\
  &= & \int_{-\infty}^{\infty} \int_{-\infty}^{\infty}
       e^{-\frac{1}{2} x^2} e^{-\frac{1}{2} y^2} dx dy \\
  &= & \int_{-\infty}^{\infty} \int_{-\infty}^{\infty}
       e^{-\frac{1}{2} (x^2 + y^2)} dx dy
\end{eqnarray*}

The exponential in this integral is begging for polar
coordinates.

\begin{eqnarray*}
x = \cos\theta & \; & y = \sin\theta \\
r^2 = x^2 + y^2 & \; & dx dy = J(r,\theta) dr d\theta
\end{eqnarray*}

$J(r,\theta)$ is the Jacobian of the transformation from
$(x,y)$ to $(r,\theta)$ coordinates.  It accounts for how
differential $dx \times dy$ area stretches to $dr \times d\theta$
area under the change of variables.

\textbf{Problem 1} Verify the value of the Jacobian.

$$
J(r,\theta) =
\begin{vmatrix}
\frac{\partial x}{\partial r} & \frac{\partial x}{\partial \theta} \\
\frac{\partial y}{\partial r} & \frac{\partial y}{\partial \theta}
\end{vmatrix} = r
$$

Substitute these back into the double integral.

$$
R^2 = \int_{0}^{2\pi} \int_{0}^{\infty}
       e^{-\frac{1}{2} r^2} r  dr d\theta \label{p1}
$$

\textbf{Problem 2} Compute this integral to determine the
normalization constant of the standard normal distribution.

When you find the value of this integral, don't forget we're
interested in $R$, not $R^2$.


\section*{Gamma Distribution}

The Gamma function is defined as

$$
\Gamma(r) = \int_0^{\infty} x^{r-1} e^{-x}dx
$$

\textbf{Problem 3} Calculate $\Gamma(1)$.

\textbf{Problem 4} Calculate $\Gamma(2)$ using integration
by parts.

\textbf{Problem 5} Show for positive integers $n > 1$ that
$\Gamma(n) = (n-1)\Gamma(n-1)$ using integration by parts. 

The formula in Problem 5 means that for positive integers,
$\Gamma(n) = (n-1)!$.  But since there is no restriction on
$r$ to be an integer, this makes the gamma function a 
continuous version of the factorial function.  Most non-integer
values have to be computed numerically.  But there is one more
special value that we can compute directly.

\textbf{Problem 6} Compute $\Gamma \left(\frac{1}{2} \right)$.
Use the substitution

$$
x = \frac{1}{2}u^2
$$

and your result from Problem 2.

The Gamma distribution has the following density function.

$$
\mbox{Gamma}(x; r, \lambda) = \frac{\lambda}{\Gamma(r)} 
                  (\lambda x)^{r-1} e^{-\lambda x}
$$

for $x \in [0, \infty)$.

\textbf{Problem 7} Show that the integral of the 
density function is 1.

$$
1 = \int_0^{\infty} \frac{\lambda}{\Gamma(r)} 
                  (\lambda x)^{r-1} e^{-\lambda x} dx
$$

\textbf{Problem 8} Show that when $r = 1$, you get the 
exponential distribution.

\textbf{Problem 9} Calculate the moment generating function
for the Gamma distribution.

$$
m_X(t) = E\left[ e^{tx} \right] = \int_0^{\infty}  e^{tx} 
    \frac{\lambda}{\Gamma(r)}   (\lambda x)^{r-1} e^{-\lambda x} dx
$$

Hint: use substituion $u=(t-\lambda)x$.

\end{document}  
